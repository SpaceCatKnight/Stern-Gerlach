\documentclass[a4paper,parskip,11pt, DIV12]{scrreprt}

\usepackage[english]{babel} % FÌr Deutsch [english] zu [ngerman] Àndern. 
\usepackage[utf8]{inputenc}
\usepackage[T1]{fontenc}
\usepackage{blindtext}
\usepackage{graphicx}
\usepackage{subfigure}
\renewcommand{\familydefault}{\sfdefault}
\usepackage{helvet}
\usepackage{fancyhdr}
\usepackage{amsmath}
\usepackage{mdwlist} %Benötigt fÌr AbstÀnde in AufzÀhlungen zu löschen
\usepackage{here}
\usepackage{calc}
\usepackage{hhline}
\usepackage{marginnote}
\usepackage{chngcntr}
\usepackage{tabularx}
\usepackage{titlesec} % TextÃŒberschriften anpassen

% \titleformat{Überschriftenklasse}[Absatzformatierung]{Textformatierung} {Nummerierung}{Abstand zwischen Nummerierung und Überschriftentext}{Code vor der Überschrift}[Code nach der Überschrift]

% \titlespacing{Überschriftenklasse}{Linker Einzug}{Platz oberhalb}{Platz unterhalb}[rechter Einzug]

\titleformat{\chapter}{\LARGE\bfseries}{\thechapter\quad}{0pt}{}
\titleformat{\section}{\Large\bfseries}{\thesection\quad}{0pt}{}
\titleformat{\subsection}{\large\bfseries}{\thesubsection\quad}{0pt}{}
\titleformat{\subsubsection}{\normalsize\bfseries}{\thesubsubsection\quad}{0pt}{}

\titlespacing{\chapter}{0pt}{-2em}{6pt}
\titlespacing{\section}{0pt}{6pt}{-0.2em}
\titlespacing{\subsection}{0pt}{5pt}{-0.4em}
\titlespacing{\subsubsection}{0pt}{-0.3em}{-1em}

%\usepackage[singlespacing]{setspace}
%\usepackage[onehalfspacing]{setspace}

\usepackage[
			%includemp,				%marginalien in Textkörper einbeziehen
			%includeall,
			%showframe,				%zeigt rahmen zum debuggen		
			marginparwidth=25mm, 	%breite der marginalien
			marginparsep=5mm,		%abstand marginalien - text
			reversemarginpar,		%marginalien links statt rechts
			%left=50mm,				%abstand von Seitenraendern
%			top=25mm,				%
%			bottom=50mm,
			]{geometry}		

%Bibliographie- Einstellungen
\usepackage[babel,german=quotes]{csquotes}
\usepackage[
   backend=bibtex8, 
   natbib=true,
    style=numeric,
    sorting=none
]{biblatex}
\bibliography{Quelle}
%Fertig Bibliographie- Einstellungen

\usepackage{hyperref}

\begin{document}

\begin{titlepage}
\begin{figure}[h]
\hfill
\subfigure{\includegraphics[scale=0.04]{uzh}}
\end{figure}
\vspace{1 cm}
\textbf{\begin{huge}Praktikumsbericht Physik 3
\end{huge}}\\
\noindent\rule{\textwidth}{1.1 pt} \\

\begin{Large}\textbf{Stern-Gerlach}
\end{Large}\\ 
\normalsize 
\par
\begingroup
\leftskip 0 cm
\rightskip\leftskip
\textbf{Module:}\\ PHY131 \\ \\
\textbf{Assistant:}\\ ???\\ \\
\textbf{Students:}\\ Manuel Sommerhalder, Stefan Hochrein, Ruben Beynon\\ \\
\textbf{Date of the experiment:}\\ 22.01.2017 \\ \\
\par
\endgroup
\clearpage



\end{titlepage}


%Start Layout
\pagestyle{fancy}
\fancyhead{} 
\fancyhead[R]{\small \leftmark}
\fancyhead[C]{\textbf{Imaging by lenses} } 
\fancyhead[L]{\includegraphics[height=2\baselineskip]{uzh}}

\fancyfoot{}
\fancyfoot[R]{\small \thepage}
\fancyfoot[L]{}
\fancyfoot[C]{}
\renewcommand{\footrulewidth}{0.4pt} 

\addtolength{\headheight}{2\baselineskip}
\addtolength{\headheight}{0.6pt}


\renewcommand{\headrulewidth}{0.6pt}
\renewcommand{\footrulewidth}{0.4pt}
\fancypagestyle{plain}{				% plain redefinieren, damit wirklich alle seiten im gleichen stil sind (ausser titlepage)
\pagestyle{fancy}}

\renewcommand{\chaptermark}[1]{ \markboth{#1}{} } %Das aktuelle Kapitel soll nicht Gross geschriben und Nummeriertwerden

\counterwithout{figure}{chapter}
\counterwithout{table}{chapter}
%Ende Layout

%\tableofcontents

\chapter{Einfuhrung}


\clearpage

\chapter{Datentabellen}

\begin{table}[H]
\centering
\renewcommand{\arraystretch}{1.2} % Abstandzwischen Zeilen
\setlength{\tabcolsep}{3mm} % Abstandzwischen Spalten
\begin{tabular}{r|r|r|r}
$I$ [mA] & $B$ [T] & $T$ [K]  & $q$ [mm]\\ \hline
402   $\pm$ 5      & 0.320   $\pm$ 0.010   & 457.3  $\pm$ 0.1   & 2.2452 $\pm$ 0.0015 \\
600    $\pm$ 5      & 0.485   $\pm$ 0.010   & 459.5  $\pm$ 0.1   &3.5437  $\pm$ 0.0025\\
700    $\pm$ 5      & 0.560   $\pm$ 0.010  & 459.7   $\pm$ 0.1   &4.2096  $\pm$ 0.0029\\
800    $\pm$ 5     & 0.635    $\pm$ 0.010 & 459.7    $\pm$ 0.1   & 4.7810 $\pm$  0.0032 \\
900    $\pm$ 5     & 0.695    $\pm$ 0.010 & 459.9    $\pm$ 0.1    &5.3034  $\pm$  0.0032 \\
1000    $\pm$ 5    & 0.750     $\pm$ 0.010 & 460.7       $\pm$ 0.1 &5.7311  $\pm$ 0.0041
\end{tabular}
\end{table} 
$L$ = 70.0 $\pm$ 0.5 $\cdot$ $10^{-3}$ m\\
$a$ = 2.5 $\pm$ 0.2 $\cdot$ $10^{-3}$ m\\
$l$ = 0.455 $\pm$ 0.001 m\\
$\epsilon$ = 0.953 $\pm$ 0.0026\\

\clearpage

\chapter{Resultate}

\begin{table}[H]
\centering
\renewcommand{\arraystretch}{1.2} % Abstandzwischen Zeilen
\setlength{\tabcolsep}{3mm} % Abstandzwischen Spalten
\begin{tabular}{l}
$m_{s,z}$ [J$\cdot$T$^{-1}$] \\ \hline
-7.904 $\pm$ 0.679 $\cdot$ $10^{-24}$\\
-8.271 $\pm$ 0.684 $\cdot$ $10^{-24}$\\
-8.513 $\pm$ 0.699 $\cdot$ $10^{-24}$\\
-8.526 $\pm$ 0.696 $\cdot$ $10^{-24}$\\
-8.645 $\pm$ 0.703 $\cdot$ $10^{-24}$\\
-8.672 $\pm$ 0.704 $\cdot$ $10^{-24}$\\
\end{tabular}
\end{table} 

$\overline{m}_{s,z}$ = -8.422 $\pm$ 0.694 $\cdot$ $10^{-24}$ J$\cdot$T$^{-1}$\\
$m_s,z$ = -9.285 $\cdot$ $10^{-24}$ J$\cdot$T$^{-1}$   \footnotesize{ \cite{https://en.wikipedia.org} }


\clearpage


\chapter{Datenanalyse}
\begin{equation}
m_{s,z}=\frac{2k_BTq}{lL(1-\frac{L}{2l})\frac{\partial B}{\partial z}}
\end{equation}

\begin{equation}
\frac{\partial B}{\partial z}=\frac{\epsilon B}{a}
\end{equation}

\begin{equation}
m_{s,z}=\frac{2k_BTqa}{lL(1-\frac{L}{2l})\epsilon B}
\end{equation}


\clearpage

\chapter{Fehlerrechnung}

\begin{equation}
m_{m_{s,z}}=\sqrt{(\frac{\partial m}{\partial q} \cdot m_q)^2+(\frac{\partial m}{\partial a} \cdot m_a)^2+(\frac{\partial m}{\partial \epsilon} \cdot m_\epsilon)^2+(\frac{\partial m}{\partial B} \cdot m_B)^2}
\end{equation}

\begin{equation}
\frac{\partial m}{\partial q} = \frac{2k_BTa}{lL(1-\frac{L}{2l})\epsilon B}
\end{equation}

\begin{equation}
\frac{\partial m}{\partial a} = \frac{2k_BTq}{lL(1-\frac{L}{2l})\epsilon B}
\end{equation}

\begin{equation}
\frac{\partial m}{\partial \epsilon} = \frac{-2k_BTaq}{lL(1-\frac{L}{2l})\epsilon^2 B}
\end{equation}

\begin{equation}
\frac{\partial m}{\partial B} = \frac{-2k_BTaq}{lL(1-\frac{L}{2l})\epsilon B^2}
\end{equation}


\clearpage
 

%\renewcommand{\bibname}{Quellenverzeichnis}
%\printbibliography

\end{document}
