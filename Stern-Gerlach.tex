\documentclass[a4paper,parskip,11pt, DIV12]{scrreprt}

\usepackage[ngerman]{babel} % FÌr Deutsch [english] zu [ngerman] Àndern. 
\usepackage[utf8]{inputenc}
\usepackage[T1]{fontenc}
\usepackage{blindtext}
\usepackage{graphicx}
\usepackage{subfigure}
\renewcommand{\familydefault}{\sfdefault}
\usepackage{helvet}
\usepackage{fancyhdr}
\usepackage{amsmath}
\usepackage{mdwlist} %Benötigt fÌr AbstÀnde in AufzÀhlungen zu löschen
\usepackage{here}
\usepackage{calc}
\usepackage{hhline}
\usepackage{marginnote}
\usepackage{chngcntr}
\usepackage{tabularx}
\usepackage{titlesec} % TextÃŒberschriften anpassen

% \titleformat{Überschriftenklasse}[Absatzformatierung]{Textformatierung} {Nummerierung}{Abstand zwischen Nummerierung und Überschriftentext}{Code vor der Überschrift}[Code nach der Überschrift]

% \titlespacing{Überschriftenklasse}{Linker Einzug}{Platz oberhalb}{Platz unterhalb}[rechter Einzug]

\titleformat{\chapter}{\LARGE\bfseries}{\thechapter\quad}{0pt}{}
\titleformat{\section}{\Large\bfseries}{\thesection\quad}{0pt}{}
\titleformat{\subsection}{\large\bfseries}{\thesubsection\quad}{0pt}{}
\titleformat{\subsubsection}{\normalsize\bfseries}{\thesubsubsection\quad}{0pt}{}

\titlespacing{\chapter}{0pt}{-2em}{6pt}
\titlespacing{\section}{0pt}{6pt}{-0.2em}
\titlespacing{\subsection}{0pt}{5pt}{-0.4em}
\titlespacing{\subsubsection}{0pt}{-0.3em}{-1em}

%\usepackage[singlespacing]{setspace}
%\usepackage[onehalfspacing]{setspace}

\usepackage[
			%includemp,				%marginalien in Textkörper einbeziehen
			%includeall,
			%showframe,				%zeigt rahmen zum debuggen		
			marginparwidth=25mm, 	%breite der marginalien
			marginparsep=5mm,		%abstand marginalien - text
			reversemarginpar,		%marginalien links statt rechts
			%left=50mm,				%abstand von Seitenraendern
%			top=25mm,				%
%			bottom=50mm,
			]{geometry}		

%Bibliographie- Einstellungen
\usepackage[babel,german=quotes]{csquotes}
\usepackage[
   backend=bibtex8, 
   natbib=true,
    style=numeric,
    sorting=none
]{biblatex}
\bibliography{Quelle}
%Fertig Bibliographie- Einstellungen

\usepackage{hyperref}

\begin{document}
\selectlanguage{ngerman}
\begin{titlepage}
\begin{figure}[h]
\hfill
\subfigure{\includegraphics[scale=0.04]{uzh}}
\end{figure}
\vspace{1 cm}
\textbf{\begin{huge}Praktikumsbericht Physik 3
\end{huge}}\\
\noindent\rule{\textwidth}{1.1 pt} \\

\begin{Large}\textbf{Stern-Gerlach}
\end{Large}\\ 
\normalsize 
\par
\begingroup
\leftskip 0 cm
\rightskip\leftskip
\textbf{Modul:}\\ PHY131 \\ \\
\textbf{Assistent:}\\ ???\\ \\
\textbf{Studenten:}\\ Manuel Sommerhalder, Stefan Hochrein, Ruben Beynon\\ \\
\textbf{Datum des Versuchs:}\\ 22.01.2017 \\ \\
\par
\endgroup
\clearpage



\end{titlepage}


%Start Layout
\pagestyle{fancy}
\fancyhead{} 
\fancyhead[R]{\small \leftmark}
\fancyhead[C]{\textbf{Stern-Gerlach} } 
\fancyhead[L]{\includegraphics[height=2\baselineskip]{uzh}}

\fancyfoot{}
\fancyfoot[R]{\small \thepage}
\fancyfoot[L]{}
\fancyfoot[C]{}
\renewcommand{\footrulewidth}{0.4pt} 

\addtolength{\headheight}{2\baselineskip}
\addtolength{\headheight}{0.6pt}


\renewcommand{\headrulewidth}{0.6pt}
\renewcommand{\footrulewidth}{0.4pt}
\fancypagestyle{plain}{				% plain redefinieren, damit wirklich alle seiten im gleichen stil sind (ausser titlepage)
\pagestyle{fancy}}

\renewcommand{\chaptermark}[1]{ \markboth{#1}{} } %Das aktuelle Kapitel soll nicht Gross geschriben und Nummeriertwerden

\counterwithout{figure}{chapter}
\counterwithout{table}{chapter}
%Ende Layout

%\tableofcontents

\chapter{Einf"uhrung}


\clearpage

\chapter{Datentabellen}

In der folgenden Tabelle sind unsere Messergebnisse dargestellt. Der Strom $I$ bezieht sich dabei auf die Magnetspulen, der Fehler darauf ist gesch"atzt und fliesst nur indirekt in die Fehlerrechnung. Das entsprechende Magnetfeld $B$ wurde mithilfe des Stroms aus der Magnetfeld-Eichkurve bestimmt, der Fehler darauf setzt sich zusammen aus dem Fehler auf dem Strom und der Ungenauigkeit des Ablesen aus der Kurve. Der Fehler auf der Temperatur ist h"oher als die Genauigkeit des Messger"ats weil die Temperatur nicht Konstant gehalten werden konnte. Der Fehler auf den Abstand der lokalen Maxima $q$ ergibt sich aus der Fit-Methode.
\begin{table}[H]
\centering
\renewcommand{\arraystretch}{1.2} % Abstandzwischen Zeilen
\setlength{\tabcolsep}{3mm} % Abstandzwischen Spalten
\begin{tabular}{r|r|r|r}
$I$ [mA] & $B$ [T] & $T$ [K]  & $q$ [mm]\\ \hline
402   $\pm$ 5      & 0.320   $\pm$ 0.010   & 457.3  $\pm$ 0.1   & 2.2452 $\pm$ 0.0015 \\
600    $\pm$ 5      & 0.485   $\pm$ 0.010   & 459.5  $\pm$ 0.1   &3.5437  $\pm$ 0.0025\\
700    $\pm$ 5      & 0.560   $\pm$ 0.010  & 459.7   $\pm$ 0.1   &4.2096  $\pm$ 0.0029\\
800    $\pm$ 5     & 0.635    $\pm$ 0.010 & 459.7    $\pm$ 0.1   & 4.7810 $\pm$  0.0032 \\
900    $\pm$ 5     & 0.695    $\pm$ 0.010 & 459.9    $\pm$ 0.1    &5.3034  $\pm$  0.0032 \\
1000    $\pm$ 5    & 0.750     $\pm$ 0.010 & 460.7       $\pm$ 0.1 &5.7311  $\pm$ 0.0041
\end{tabular}
\caption[Daten]{Gemessene Daten}
\end{table} 
F"ur die Berechnung des magnetischen Moments des Elektrons haben wir die folgenden Konstanten ben"utzt:\\


L"ange der Pole:\\
$L$ = 70.0 $\pm$ 0.5 $\cdot$ $10^{-3}$ m\\
Radius des konvexen Pols:\\
$a$ = 2.5 $\pm$ 0.2 $\cdot$ $10^{-3}$ m\\
Distanz Magnetfeldeintritt-Detektor :\\
$l$ = 0.455 $\pm$ 0.001 m\\
Proportionalit"ats Faktor zwischen dem Magnetfeld und seiner "Anderung in z-Richtung:
$\epsilon$ = 0.953 $\pm$ 0.0026\\


Die Fehler auf den L"angen $L$, $l$, $a$ haben wir anhand ihrer Gr"osse und der Schwierigkeit der Durchf"uhrung der Messung gesch"atzt. Der Fehler aus $\epsilon$ ist aus den Praktikumsunterlagen bekannt.
\clearpage

\chapter{Resultate}

In der Folgenden Tabelle sind unsere berechneten Werte f"ur das magnetische Moment des Elektrons $m_{s,z}$ zu verschieden starken Ablenkungen. Dabei ist Augenf"allig, dass unser Wert mit st"arkerer Ablenkung steigt. 

\begin{table}[H]
\centering
\renewcommand{\arraystretch}{1.2} % Abstandzwischen Zeilen
\setlength{\tabcolsep}{3mm} % Abstandzwischen Spalten
\begin{tabular}{c}
$m_{s,z}$ [J$\cdot$T$^{-1}$] \\ \hline
-7.904 $\pm$ 0.679 $\cdot$ $10^{-24}$\\
-8.271 $\pm$ 0.684 $\cdot$ $10^{-24}$\\
-8.513 $\pm$ 0.699 $\cdot$ $10^{-24}$\\
-8.526 $\pm$ 0.696 $\cdot$ $10^{-24}$\\
-8.645 $\pm$ 0.703 $\cdot$ $10^{-24}$\\
-8.672 $\pm$ 0.704 $\cdot$ $10^{-24}$\\
\end{tabular}
\end{table} 

Der daraus errechnete Mittelwert ist:\\
\begin{center}
\begin{large}
 $\overline{m}_{s,z}$ = -8.422 $\pm$ 0.694 $\cdot$ $10^{-24}$ J$\cdot$T$^{-1}$\\
 \end{large}
\end{center}

Wenn wir dies mit dem Literaturen vergleichen $m_{s,z}$ = -9.285 $\cdot$ $10^{-24}$ J$\cdot$T$^{-1}$   (Quelle: {\footnotesize \cite{https://en.wikipedia.org} }) so sehen wir, dass der Literaturwert knapp nicht in der Fehlergrenze unseres Wertes liegt. M"ogliche Fehlerquellen sehen wir in der unstetigen Temperatur des Ofens und durch eventuell schlecht kalibrierte Ger"ate. Nichts desto trotz hat sich im Experiment die Aufspaltung in zwei Maxima gezeigt und somit die zweim"oglichen Spinneinstellungen gezeigt.


\clearpage


\chapter{Datenanalyse}\label{Datenanalyse}

Aus "Uberlegungen zum geometrischen Aufbau der Versuchsanlage und der Kraft eines r"aumlich variierenden Magnetfeldes auf ein magnetisches Moment $\vec{ \mu }$ ergibt sich Formel (4.1), welche wir aus der Versuchsanleitung "ubernommen haben. Dabei sind $u_e$ die wahrscheinlichsten Aufprallpunkte der Atome und $q$ ist der Abstand der beiden Maxima, die sich aus den zwei Spinneinstellungen ergeben.


\begin{equation}
u_e=\pm \frac{q}{3}=\frac{lL(1-\frac{L}{2l})m_{s,z} \frac{\partial B}{\partial z}}{6k_BT}
\end{equation}

Die Ver"anderung des Magnetfeld in $z$-Richtung ergibt sich aus der Geometrie der Polschuhe, dabei ist der Aufbau des Magneten so gew"ahlt, dass sich ein m"oglichst konstanter Gradient in $z$-Richtung entlang des Atomstrahls ergibt. Aufgrund der Tatsache, dass der Gradient eines Magnetfeld schwer zu messen ist wurde die Annahme getroffen, dass $\frac{ \partial B}{ \partial z}$ proportional zu $B$ ist f"ur kleine Auslenkungen in $z$. Daraus ergibt sich der Proportionalit"atsfaktor $\epsilon$.

\begin{equation}
\frac{\partial B}{\partial z}=\frac{\epsilon B}{a}
\end{equation}

Daraus ergibt sich folgende Formel zur Berechnung des magnetischen Moments des Elektrons $m_{s,z}$:
\begin{equation}
m_{s,z}=\frac{2k_BTqa}{lL(1-\frac{L}{2l})\epsilon B}
\end{equation}


\clearpage

\chapter{Fehlerrechnung}
Die Fehlerfortpflanzung wurde nach der Gaussschen Formel (5.1) ausgef"uhrt. Dabei haben wir nur die Fehler auf $q$, $a$, $\epsilon$ $\&$ $B$ in unsere Berechnung einbezogen. Die Fehler auf der Temperatur $T$ und den Aufbaukonstanten $L$ $\&$ $l$ haben wir vernachl"assigt, dies weil eine Verdoppelung des Fehlers nur eine Ver"anderung in der dritten Nachkommastelle des Gesammtfehlers zur Folge hatte.


\begin{equation}
m_{m_{s,z}}=\sqrt{(\frac{\partial m}{\partial q} \cdot m_q)^2+(\frac{\partial m}{\partial a} \cdot m_a)^2+(\frac{\partial m}{\partial \epsilon} \cdot m_\epsilon)^2+(\frac{\partial m}{\partial B} \cdot m_B)^2}
\end{equation}

Die folgenden vier Gleichungen sind die Partiellen Ableitungen der Gleichung (4.1), welche wir zur Berechnung des Gesamtfehlers auf $m_{s,z}$ ben"otigt haben.

\begin{equation}
\frac{\partial m}{\partial q} = \frac{2k_BTa}{lL(1-\frac{L}{2l})\epsilon B}
\end{equation}

\begin{equation}
\frac{\partial m}{\partial a} = \frac{2k_BTq}{lL(1-\frac{L}{2l})\epsilon B}
\end{equation}

\begin{equation}
\frac{\partial m}{\partial \epsilon} = \frac{-2k_BTaq}{lL(1-\frac{L}{2l})\epsilon^2 B}
\end{equation}

\begin{equation}
\frac{\partial m}{\partial B} = \frac{-2k_BTaq}{lL(1-\frac{L}{2l})\epsilon B^2}
\end{equation}


\clearpage
 

%\renewcommand{\bibname}{Quellenverzeichnis}
%\printbibliography

\end{document}
